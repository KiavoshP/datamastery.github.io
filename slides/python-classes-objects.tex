% Created 2018-02-12 Mon 13:40
\documentclass[smaller]{beamer}
\usepackage[utf8]{inputenc}
\usepackage[T1]{fontenc}
\usepackage{fixltx2e}
\usepackage{graphicx}
\usepackage{longtable}
\usepackage{float}
\usepackage{wrapfig}
\usepackage{rotating}
\usepackage[normalem]{ulem}
\usepackage{amsmath}
\usepackage{textcomp}
\usepackage{marvosym}
\usepackage{wasysym}
\usepackage{amssymb}
\usepackage{hyperref}
\tolerance=1000
\usepackage{verbatim, multicol, tabularx,color}
\usepackage{amsmath,amsthm, amssymb, latexsym, listings, qtree}
\lstset{frame=tb, aboveskip=1mm, belowskip=0mm, showstringspaces=false, columns=flexible, basicstyle={\scriptsize\ttfamily}, numbers=left, frame=single, breaklines=true, breakatwhitespace=true, keywordstyle=\bf}
\setbeamertemplate{footline}[frame number]
\hypersetup{colorlinks=true,urlcolor=blue}
\logo{\includegraphics[height=.75cm]{GeorgiaTechLogo-black-gold.png}}
\usetheme{default}
\author{}
\date{}
\title{Classes and Objects}
\hypersetup{
  pdfkeywords={},
  pdfsubject={},
  pdfcreator={Emacs 25.3.3 (Org mode 8.2.10)}}
\begin{document}

\maketitle

\section{Classes and Objects}
\label{sec-1}

\begin{frame}[fragile,label=sec-1-1]{Python is Object-Oriented}
 Every value in Python is an object, meaning an instance of a class. Even values that are considered "primitive" in some other languages.

\lstset{language=Python,label= ,caption= ,numbers=none}
\begin{lstlisting}
>>> type(1)
<class 'int'>
\end{lstlisting}
\end{frame}


\begin{frame}[fragile,label=sec-1-2]{Class Definitions}
 \lstset{language=Python,label= ,caption= ,numbers=none}
\begin{lstlisting}
class <class_name>(<superclasses>):
    <body>
\end{lstlisting}

\begin{itemize}
\item \verb~<class_name>~ is an identifier
\item \verb~<superclasses>~ is a comma-separated list of superclasses. Can be empty, in which case \verb~object~ is implicit superclass
\item \verb~<body>~ is a non-empty sequence of statements
\end{itemize}

A class definition creates a class object in much the same way that a function definition creates a function object.
\end{frame}

\begin{frame}[fragile,label=sec-1-3]{Class Attributes}
 \lstset{language=Python,label= ,caption= ,numbers=none}
\begin{lstlisting}
class Stark:
    creator = "George R.R. Martin"
    words = "Winter is coming"
    sigil = "Direwolf"
    home = "Winterfell"

    def __init__(self, name=None):
        self.name = name if name else "No one"

    def full_name(self):
        return "{} Stark".format(self.name)
\end{lstlisting}

\verb~creator~, \verb~words~, \verb~sigil~, and \verb~home~ are \alert{class attributes}. Class attributes belong to the class and are shared by all instances
\end{frame}

\begin{frame}[fragile,label=sec-1-4]{Instance Attributes}
 \lstset{language=Python,label= ,caption= ,numbers=none}
\begin{lstlisting}
class Stark:
    creator = "George R.R. Martin"
    words = "Winter is coming"
    sigil = "Direwolf"
    home = "Winterfell"

    def __init__(self, name=None):
        self.name = name if name else "No one"

    def full_name(self):
        return "{} Stark".format(self.name)
\end{lstlisting}

\begin{itemize}
\item \verb~self.name~ is an instance attribute becuase it is prefaced with \verb~self.~ and defined in a method that has a first parameter named \verb~self~. Each instance of the class has its own copies of instance attributes.
\item \verb~full_name~ is an instance method because it defined in a class and has at least one parameter. The first parameter is implicitly a reference to the instance on which a method is called.
\end{itemize}
\end{frame}

\begin{frame}[fragile,label=sec-1-5]{Classes and Objects}
 In this example, \verb~ned~ and \verb~robb~ are \alert{instances} of \verb~Stark~. Each instance has it's own \verb~name~.

\lstset{language=Python,label= ,caption= ,numbers=none}
\begin{lstlisting}
>>> import got
>>> ned = got.Stark("Eddard")
>>> ned.name
'Eddard'
>>> robb = got.Stark("Robb")
>>> robb.name
'Robb'
\end{lstlisting}

Ivoking the \verb~full_name()~ method on an object implicitly passes the object as the first argument (\verb~self~), which you could (but shoudn't) do explicitly:

\lstset{language=Python,label= ,caption= ,numbers=none}
\begin{lstlisting}
>>> ned.full_name()          # This is normal
'Eddard Stark'
>>> got.Stark.full_name(ned) # This is only instructive
'Eddard Stark'
\end{lstlisting}
\end{frame}

\begin{frame}[fragile,label=sec-1-6]{Class Members}
 Each instance shares the class attributes \verb~creator~, \verb~words~, \verb~sigil~, and \verb~home~.

\lstset{language=Python,label= ,caption= ,numbers=none}
\begin{lstlisting}
>>> got.Stark.sigil
'Direwolf'
>>> ned.sigil
'Direwolf'
>>> robb.sigil
'Direwolf'
\end{lstlisting}

Remember that the \verb~is~ operator returns \verb~True~ if its operands reference the same object in memory. So this deomonstrates that \verb~sigil~ is shared between the \verb~Stark~ class and all instances of the \verb~Stark~ class:

\lstset{language=Python,label= ,caption= ,numbers=none}
\begin{lstlisting}
>>> got.Stark.sigil is ned.sigil
True
\end{lstlisting}
\end{frame}


\begin{frame}[fragile,label=sec-1-7]{Superclasses}
 Superclasses, or parent classes, or base classes, define attributes that you wish to be common to a family of objects.

Notice that all of our noble houses have the same creator, and every instance has a name. We can represent this commonality by creating a base class for all house classes:

\lstset{language=Python,label= ,caption= ,numbers=none}
\begin{lstlisting}
class GotCharacter:
    creator = "George R.R. Martin"

    def __init__(self, name=None):
        self.name = name if name else "No one"
\end{lstlisting}
\end{frame}

\begin{frame}[fragile,label=sec-1-8]{Refactored \verb~Stark~}
 Here is \verb~Stark~ refactored to use the \verb~GotCharacter~ superclass:

\lstset{language=Python,label= ,caption= ,numbers=none}
\begin{lstlisting}
class Stark(GotCharacter):
    words = "Winter is coming"
    sigil = "Direwolf"
    home = "Winterfell"

    def __init__(self, name):
        # This is how you invoke a superclass method
        super().__init__(name)
\end{lstlisting}

Exercise: refactor the other GoT houses to use the \verb~GotCharacter~ superclass.
\end{frame}

\begin{frame}[fragile,label=sec-1-9]{Magic, a.k.a., Dunder Methods}
 Methods with names that begin and end with \verb~__~

\lstset{language=Python,label= ,caption= ,numbers=none}
\begin{lstlisting}
class SuperTrooper(Trooper):

    def __init__(self, name, is_mustached):
        super().__init__(name)
        self.is_mustached = is_mustached

    # Used by print()
    def __str__(self):
        return "<{} {}>".format(self.name, ":-{" if self.is_mustached else ":-|")

    # Used by REPL
    def __repr__(self):
        return str(self)

    # Makes instances of SuperTrooper orderable
    def __lt__(self, other):
        if self.is_mustached and not other.is_mustached:
            return False
        elif not self.is_mustached and other.is_mustached:
            return True
        else:
            return self.name < other.name
\end{lstlisting}
\end{frame}

\begin{frame}[fragile,label=sec-1-10]{Sortable SuperTroopers}
 With the definition of \verb~__lt__(self, other)~ in \verb~SuperTrooper~, a list of \verb~SuperTrooper~ is sortable.

\lstset{language=Python,label= ,caption= ,numbers=none}
\begin{lstlisting}
sts = [SuperTrooper("Thorny", True),
       SuperTrooper("Mac", True),
       SuperTrooper("Rabbit", True),
       SuperTrooper("Farva", True),
       SuperTrooper("Foster", False)]
print("SuperTroopers:")
print(sts)
print("SuperTroopers sorted by mustache, then by name:")
print(sorted(sts))
\end{lstlisting}

Produces:

\lstset{language=sh,label= ,caption= ,numbers=none}
\begin{lstlisting}
SuperTroopers:
[<Thorny :-{>, <Mac :-{>, <Rabbit :-{>, <Farva :-{>, <Foster :-|>]
SuperTroopers sorted by mustache, then by name:
[<Foster :-|>, <Farva :-{>, <Mac :-{>, <Rabbit :-{>, <Thorny :-{>]
\end{lstlisting}
\end{frame}

\begin{frame}[fragile,label=sec-1-11]{Final Thoughts}
 Recall the design of the Game of Thrones character types:

\begin{itemize}
\item A superclass \verb~GotCharacter~ with class attributes common to Got characters of all houses.
\item A class for each house, subclassing \verb~GotCharacter~ and defining the common attributes of all house members.
\begin{itemize}
\item Each character is an instance of one of these house classes, like \verb~Lannister~, \verb~Stark~, etc.
\end{itemize}
\end{itemize}

Is this a good design? What if you had an instance of a \verb~Stark~ and you later found out that they're a \verb~Targaryen~? Refactor the design of the Got character classes to allow a character to change houses without having to modify the code and re-run the program.
\end{frame}


\begin{frame}[label=sec-1-12]{Conclusion}
\href{https://www.youtube.com/embed/az5qOjhsang}{Magic!}
\end{frame}
% Emacs 25.3.3 (Org mode 8.2.10)
\end{document}
