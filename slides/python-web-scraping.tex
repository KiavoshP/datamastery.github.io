% Created 2018-02-28 Wed 13:35
\documentclass[smaller]{beamer}
\usepackage[utf8]{inputenc}
\usepackage[T1]{fontenc}
\usepackage{fixltx2e}
\usepackage{graphicx}
\usepackage{longtable}
\usepackage{float}
\usepackage{wrapfig}
\usepackage{rotating}
\usepackage[normalem]{ulem}
\usepackage{amsmath}
\usepackage{textcomp}
\usepackage{marvosym}
\usepackage{wasysym}
\usepackage{amssymb}
\usepackage{hyperref}
\tolerance=1000
\usepackage{verbatim, multicol, tabularx,color}
\usepackage{amsmath,amsthm, amssymb, latexsym, listings, qtree}
\lstset{frame=tb, aboveskip=1mm, belowskip=0mm, showstringspaces=false, columns=flexible, basicstyle={\scriptsize\ttfamily}, numbers=left, frame=single, breaklines=true, breakatwhitespace=true, keywordstyle=\bf}
\setbeamertemplate{footline}[frame number]
\hypersetup{colorlinks=true,urlcolor=blue}
\logo{\includegraphics[height=.75cm]{GeorgiaTechLogo-black-gold.png}}
\usetheme{default}
\author{}
\date{}
\title{Web Scraping}
\hypersetup{
  pdfkeywords={},
  pdfsubject={},
  pdfcreator={Emacs 25.3.3 (Org mode 8.2.10)}}
\begin{document}

\maketitle

\section{Web Scraping}
\label{sec-1}

\begin{frame}[label=sec-1-1]{Web Scraping}
Two ways to mine data from the web

\begin{itemize}
\item The hard way, by web scraping
\item The easy way, using web service APIs
\end{itemize}

We’ll see examples of both.
\end{frame}

\begin{frame}[label=sec-1-2]{Web Scraping}
Web scraping, a.k.a. screen scraping, means getting data from a web page. Suppose we want to get the current wind data for a city from \href{http://openweathermap.org/}{Open Weather Map}.

\includegraphics[width=.9\linewidth]{weather-city.png}
\end{frame}

\begin{frame}[label=sec-1-3]{What is a Web Page?}
A web page is a chunk of text containing HTML code. The browser "renders" the HTML graphically. So web scraping means analyzing text using Python's text processing features.

\includegraphics[width=.9\linewidth]{weather-city-html.png}
\end{frame}


\begin{frame}[label=sec-1-4]{Finding The Data On the Page}
First you need to find the data within the HTML code for a page so you can construct a regex. Your browser's developer features can help you find the data:

\includegraphics[width=.9\linewidth]{weather-city-devtools.png}
\end{frame}


\begin{frame}[fragile,label=sec-1-5]{Getting the Web Page's HTML Code}
 To get the HTML code of the web page into a Python string variable that you can play with, use Python's \verb~urllib.request~ module:

\lstset{language=Python,label= ,caption= ,numbers=none}
\begin{lstlisting}
import urllib.request
# 2994160 is the city code for Metz, FR
request = urllib.request.Request("http://www.openweathermap.com/city/2994160")
response = urllib.request.urlopen(request)
page_bytes = response.read()
page_text = page_bytes.decode()
# page_text is Python str containing the HTML code
\end{lstlisting}

or with the \href{http://docs.python-requests.org/en/master/}{requests} module, which is what we'll use:

\lstset{language=Python,label= ,caption= ,numbers=none}
\begin{lstlisting}
import requests
resp = requests.get("http://www.openweathermap.com/city/2994160")
resp.text # the text of the web page
\end{lstlisting}
\end{frame}

\begin{frame}[fragile,label=sec-1-6]{Extracting the Data}
 Looks like the wind data is in the second \verb~<td>~ element after the \verb~<div class="weather-widget">~ tag, following a \verb~<td>Wind</td>~ element. We can play around with the HTML text in the Python REPL. We eventually end up with:

\lstset{language=Python,label= ,caption= ,numbers=none}
\begin{lstlisting}
wind = re.findall(r'<td>Wind</td><td>(.+?)</td>', page_text.replace("\n",""))[0]
\end{lstlisting}

Notice that we used a capture group to get the element data.
\end{frame}

\begin{frame}[fragile,label=sec-1-7]{Aside: Parsing HTML}
 HTML is context free language, which roughly means that it supports arbitrary nesting of elements. For example, you could have arbitrarily nested \verb~div~ elements with "leaf" elements containing text data, e.g.:

\lstset{language=HTML,label= ,caption= ,numbers=none}
\begin{lstlisting}
<div>
    <div>
        <div>some text</div>
    </div>
</div>
\end{lstlisting}

By the rules of HTML, you could nest \verb~div~ tags as deeply as you want.

Regular expressions match regular laguages, which don't support arbitrary nesting. So how can we use regexes to "parse" HTML?
\end{frame}

\begin{frame}[label=sec-1-8]{Regex Matching in HTML Code}
Parsing means scanning the linear sequence of symbols in a string to determine its structure (usually by putting the symbols in a tree).

We don't need to parse HTML to find data on a web page. While the HTML \alert{language} supports arbitrary nesting, a particular web page will be nested to a particular depth, resulting a simple linear sequence of symbols that we can match with a regular expression.
\end{frame}
% Emacs 25.3.3 (Org mode 8.2.10)
\end{document}
